\documentclass{article}

\setlength{\oddsidemargin}{0 in}
\setlength{\topmargin}{-0.6 in}
\setlength{\textwidth}{6.5 in}
\setlength{\textheight}{8.5 in}
\setlength{\headsep}{0.75 in}
\setlength{\parindent}{0 in}
\setlength{\parskip}{0.1 in}

%

\usepackage{amsmath,amsfonts,graphicx}
\usepackage{amssymb}
\usepackage[utf8]{inputenc}
\usepackage[english]{babel}
\usepackage{enumitem}
\usepackage{graphicx}
\graphicspath{ {images/} }
\usepackage{amsthm}
\usepackage{mathtools}
\usepackage{hyperref}
\usepackage{verbatim}
\usepackage{bbm}
\usepackage{bm}
\usepackage{pgfplots}
\hypersetup{
    colorlinks,
    citecolor=blue,
    filecolor=blue,
    linkcolor=blue,
    urlcolor=blue
}
\usepackage{pgfplots}

%

\makeatletter
\newcommand*\dotp{\mathpalette\dotp@{.5}}
\newcommand*\dotp@[2]{\mathbin{\vcenter{\hbox{\scalebox{#2}{$\m@th#1\bullet$}}}}}
\makeatother

\makeatletter
\newsavebox\myboxA
\newsavebox\myboxB
\newlength\mylenA

\newcommand*\xoverline[2][0.75]{%
    \sbox{\myboxA}{$\m@th#2$}%
    \setbox\myboxB\null% Phantom box
    \ht\myboxB=\ht\myboxA%
    \dp\myboxB=\dp\myboxA%
    \wd\myboxB=#1\wd\myboxA% Scale phantom
    \sbox\myboxB{$\m@th\overline{\copy\myboxB}$}%  Overlined phantom
    \setlength\mylenA{\the\wd\myboxA}%   calc width diff
    \addtolength\mylenA{-\the\wd\myboxB}%
    \ifdim\wd\myboxB<\wd\myboxA%
       \rlap{\hskip 0.5\mylenA\usebox\myboxB}{\usebox\myboxA}%
    \else
        \hskip -0.5\mylenA\rlap{\usebox\myboxA}{\hskip 0.5\mylenA\usebox\myboxB}%
    \fi}
\makeatother

%

\newcounter{lecnum}
\renewcommand{\thepage}{\thelecnum-\arabic{page}}
\renewcommand{\thesection}{\thelecnum.\arabic{section}}
\renewcommand{\theequation}{\thelecnum.\arabic{equation}}
\renewcommand{\thefigure}{\thelecnum.\arabic{figure}}
\renewcommand{\thetable}{\thelecnum.\arabic{table}}

%
% The following macro is used to generate the header.
%
\newcommand{\lecture}[4]{
   \pagestyle{myheadings}
   \thispagestyle{plain}
   \newpage
   \setcounter{lecnum}{#1}
   \setcounter{page}{1}
   \noindent
   \begin{center}
   \framebox{
      \vbox{\vspace{2mm}
    \hbox to 6.28in { {\bf Math 170B
	\hfill Winter 2017} }
       \vspace{4mm}
       \hbox to 6.28in { {\Large \hfill Problem Set #1 \hfill} }
       \vspace{2mm}
       \hbox to 6.28in { {\it Lecturer: #3 \hfill #4} }
      \vspace{2mm}}
   }
   \end{center}
   \markboth{Problem Set #1}{Problem Set #1}
  }

\renewcommand{\cite}[1]{[#1]}
\def\beginrefs{\begin{list}%
        {[\arabic{equation}]}{\usecounter{equation}
         \setlength{\leftmargin}{2.0truecm}\setlength{\labelsep}{0.4truecm}%
         \setlength{\labelwidth}{1.6truecm}}}
\def\endrefs{\end{list}}
\def\bibentry#1{\item[\hbox{[#1]}]}


\newcommand{\fig}[3]{
			\vspace{#2}
			\begin{center}
			Figure \thelecnum.#1:~#3
			\end{center}
	}
% Use these for theorems, lemmas, proofs, etc.
\newtheoremstyle{break}
  {\topsep}{\topsep}%
  {\itshape}{}%
  {\bfseries}{}%
  {\newline}{}%
\theoremstyle{break}
\newtheorem{theorem}{Theorem}[lecnum]
\newtheorem{lemma}[theorem]{Lemma}
\newtheorem{lemma*}{Lemma}
\newtheorem{problem}{Problem}
\newtheorem{proposition}[theorem]{Proposition}
\newtheorem{claim}[theorem]{Claim}
\newtheorem{corollary}[theorem]{Corollary}
\newtheorem{definition}[theorem]{Definition}
\newtheorem{example}{Example}

\newenvironment{solution}{{\bf Solution:}}{\hfill\rule{2mm}{2mm}}
\renewenvironment{proof}{{\bf Proof:}}{\hfill\rule{2mm}{2mm}}

\newcommand{\E}{\mathrm{E}}
\newcommand{\var}{\mathrm{Var}}
\newcommand{\cov}{\mathrm{Cov}}
\newcommand{\N}{\mathbb{N}}
\newcommand{\R}{\mathbb{R}}
\newcommand{\Z}{\mathbb{Z}}
\newcommand{\Q}{\mathbb{Q}}
\newcommand{\C}{\mathbb{C}}
\newcommand{\X}{\mathbb{X}}
\renewcommand{\L}{\mathcal{L}}
\renewcommand{\P}{\mathbf{P}}
\newcommand{\B}{\mathcal{B}}
\newcommand{\interior}{\text{int}}
\newcommand{\exterior}{\text{ext}}
\newcommand{\bigci}{\mathrel{\text{\scalebox{1.07}{$\perp\mkern-10mu\perp$}}}}
\newcommand{\nsum}{\sum_{n=0}^{\infty}}

\setlist[itemize]{topsep=0pt}
\setlist[enumerate]{topsep=0pt}

\begin{document}

\lecture{5}{}{Steven Heilman}{Kyle Barron}



\subsection*{Exercise 1}


Let $X$ be a standard Gaussian random variable.  Let $t>0$ and let $n$ be a positive even integer.  Show that
$$\P(X>t)\leq\frac{(n-1)(n-3)\cdots(3)(1)}{t^{n}}.$$
That is, the function $t\mapsto\P(X>t)$ decays faster than any monomial.


\subsection*{Exercise 2}


Let $X$ be a random variable.  Let $t>0$.  Show that
$$\P(|X|>t)\leq\frac{\E X^{4}}{t^{4}}.$$


\subsection*{Exercise 3}
(The Chernoff Bound.)
Let $X$ be a random variable and let $r>0$.  Show that, for any $t>0$,
$$\P(X>r)\leq e^{-tr}M_{X}(t).$$
Consequently, if $X_{1},\ldots,X_{n}$ are independent random variables with the same CDF, and if $r,t>0$,
$$\P\left(\frac{1}{n}\sum_{i=1}^{n}X_{i}>r\right)\leq e^{-trn}(M_{X_{1}}(t))^{n}.$$
For example, if $X_{1},\ldots,X_{n}$ are independent Bernoulli random variables with parameter $0<p<1$,  and if $r,t>0$,
$$\P\left(|\frac{X_{1}+\cdots+X_{n}}{n}-p|>r\right)\leq e^{-trn}(p(1-p)(1+e^{t}))^{n}.$$
And if we choose $t=1/10$, then the quantity $\P\left(\frac{1}{n}|\sum_{i=1}^{n}(X_{i}-p)|>r\right)$ becomes exponentially small as either $n$ or $r$ become large.  That is, $\frac{1}{n}\sum_{i=1}^{n}X_{i}$ becomes very close to its mean.  Importantly, the Chernoff bound is much stronger than either Markov's or Cheyshev's inequality, since they only respectively imply that
$$\P\left(|\frac{X_{1}+\cdots+X_{n}}{n}-p|>r\right)\leq \frac{2p(1-p)}{nr},   % E|X-p| = p (1-p) + (1-p)p=2p(1-p)
\quad\P\left(|\frac{X_{1}+\cdots+X_{n}}{n}-p|>r\right)\leq \frac{p(1-p)}{nr^{2}}.$$ % var(X-p)= E(X-p)^2 = p(1-p)^2+(1-p)p^2= p(1-p)[1-p+p]=p(1-p)
%In particular, for any $r>0$,
%$$\lim_{n\to\infty}\P\left(\abs{\frac{X_{1}+\cdots+X_{n}}{n}-p}>r\right)=0$$
%
%what is M_X-p for Bernoulli X?  It is 1*(-p)(1-p)+[e^t](1-p)*p=p(1-p)[-1+e^t]
%  what is M_{|X-p|}?  it is p*(1-p)+e^t p(1-p)= p(1-p)[1+e^t]


\subsection*{Exercise 4}

Let $X_{1},X_{2},\ldots$ be independent random variables, each with exponential distribution with parameter $\lambda=1$.  For any $n \geq 1$, let $Y_{n}\coloneqq\max(X_{1},\ldots,X_{n})$.  Let $0<a<1<b$.  Show that $\P(Y_{n}\leq a\log n)\to0$ as $n\to\infty$, and $\P(Y_{n}\leq b\log n)\to1$ as $n\to\infty$.  Conclude that $Y_{n}/\log n$ converges to $1$ in probability as $n\to\infty$.


\subsection*{Exercise 5}

We say that random variables $X_{1},X_{2},\ldots$ converge to a random variable $X$ in $L_{2}$ if
$$\lim_{n\to\infty}\E|X_{n}-X|^{2}=0.$$
Show that, if $X_{1},X_{2},\ldots$ converge to $X$ in $L_{2}$, then $X_{1},X_{2},\ldots$ converges to $X$ in probability.

Is the converse true?  Prove your assertion.



\subsection*{Exercise 6}


Let $X_{1},X_{2},\ldots$ be independent, identically distributed random variables such that $\E|X|<\infty$ and $\mathrm{var}(X)<\infty$.  For any $n\geq1$, define
$$Y_{n}\coloneqq\frac{1}{n}\sum_{i=1}^{n}X_{i}^{2}.$$
Show that $Y_{1},Y_{2},\ldots$ converges in probability.  Express the limit in terms of $\E X$ and $\mathrm{var}(X)$.









\end{document}
