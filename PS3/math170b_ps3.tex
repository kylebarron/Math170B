\documentclass{article}

\setlength{\oddsidemargin}{0 in}
\setlength{\topmargin}{-0.6 in}
\setlength{\textwidth}{6.5 in}
\setlength{\textheight}{8.5 in}
\setlength{\headsep}{0.75 in}
\setlength{\parindent}{0 in}
\setlength{\parskip}{0.1 in}

%

\usepackage{amsmath,amsfonts,graphicx}
\usepackage{amssymb}
\usepackage[utf8]{inputenc}
\usepackage[english]{babel}
\usepackage{enumitem}
\usepackage{graphicx}
\graphicspath{ {images/} }
\usepackage{amsthm}
\usepackage{mathtools}
\usepackage{hyperref}
\usepackage{verbatim}
\usepackage{bbm}
\usepackage{bm}
\usepackage{pgfplots}
\hypersetup{
    colorlinks,
    citecolor=blue,
    filecolor=blue,
    linkcolor=blue,
    urlcolor=blue
}
\usepackage{pgfplots}

%

\makeatletter
\newcommand*\dotp{\mathpalette\dotp@{.5}}
\newcommand*\dotp@[2]{\mathbin{\vcenter{\hbox{\scalebox{#2}{$\m@th#1\bullet$}}}}}
\makeatother

\makeatletter
\newsavebox\myboxA
\newsavebox\myboxB
\newlength\mylenA

\newcommand*\xoverline[2][0.75]{%
    \sbox{\myboxA}{$\m@th#2$}%
    \setbox\myboxB\null% Phantom box
    \ht\myboxB=\ht\myboxA%
    \dp\myboxB=\dp\myboxA%
    \wd\myboxB=#1\wd\myboxA% Scale phantom
    \sbox\myboxB{$\m@th\overline{\copy\myboxB}$}%  Overlined phantom
    \setlength\mylenA{\the\wd\myboxA}%   calc width diff
    \addtolength\mylenA{-\the\wd\myboxB}%
    \ifdim\wd\myboxB<\wd\myboxA%
       \rlap{\hskip 0.5\mylenA\usebox\myboxB}{\usebox\myboxA}%
    \else
        \hskip -0.5\mylenA\rlap{\usebox\myboxA}{\hskip 0.5\mylenA\usebox\myboxB}%
    \fi}
\makeatother

%

\newcounter{lecnum}
\renewcommand{\thepage}{\thelecnum-\arabic{page}}
\renewcommand{\thesection}{\thelecnum.\arabic{section}}
\renewcommand{\theequation}{\thelecnum.\arabic{equation}}
\renewcommand{\thefigure}{\thelecnum.\arabic{figure}}
\renewcommand{\thetable}{\thelecnum.\arabic{table}}

%
% The following macro is used to generate the header.
%
\newcommand{\lecture}[4]{
   \pagestyle{myheadings}
   \thispagestyle{plain}
   \newpage
   \setcounter{lecnum}{#1}
   \setcounter{page}{1}
   \noindent
   \begin{center}
   \framebox{
      \vbox{\vspace{2mm}
    \hbox to 6.28in { {\bf Math 170B
	\hfill Winter 2017} }
       \vspace{4mm}
       \hbox to 6.28in { {\Large \hfill Problem Set #1 \hfill} }
       \vspace{2mm}
       \hbox to 6.28in { {\it Lecturer: #3 \hfill #4} }
      \vspace{2mm}}
   }
   \end{center}
   \markboth{Problem Set #1}{Problem Set #1}
  }

\renewcommand{\cite}[1]{[#1]}
\def\beginrefs{\begin{list}%
        {[\arabic{equation}]}{\usecounter{equation}
         \setlength{\leftmargin}{2.0truecm}\setlength{\labelsep}{0.4truecm}%
         \setlength{\labelwidth}{1.6truecm}}}
\def\endrefs{\end{list}}
\def\bibentry#1{\item[\hbox{[#1]}]}


\newcommand{\fig}[3]{
			\vspace{#2}
			\begin{center}
			Figure \thelecnum.#1:~#3
			\end{center}
	}
% Use these for theorems, lemmas, proofs, etc.
\newtheoremstyle{break}
  {\topsep}{\topsep}%
  {\itshape}{}%
  {\bfseries}{}%
  {\newline}{}%
\theoremstyle{break}
\newtheorem{theorem}{Theorem}[lecnum]
\newtheorem{lemma}[theorem]{Lemma}
\newtheorem{lemma*}{Lemma}
\newtheorem{problem}{Problem}
\newtheorem{proposition}[theorem]{Proposition}
\newtheorem{claim}[theorem]{Claim}
\newtheorem{corollary}[theorem]{Corollary}
\newtheorem{definition}[theorem]{Definition}
\newtheorem{example}{Example}

\newenvironment{solution}{{\bf Solution:}}{\hfill\rule{2mm}{2mm}}
\renewenvironment{proof}{{\bf Proof:}}{\hfill\rule{2mm}{2mm}}

\newcommand{\E}{\mathrm{E}}
\newcommand{\var}{\mathrm{Var}}
\newcommand{\cov}{\mathrm{Cov}}
\newcommand{\N}{\mathbb{N}}
\newcommand{\R}{\mathbb{R}}
\newcommand{\Z}{\mathbb{Z}}
\newcommand{\Q}{\mathbb{Q}}
\newcommand{\C}{\mathbb{C}}
\newcommand{\X}{\mathbb{X}}
\renewcommand{\L}{\mathcal{L}}
\renewcommand{\P}{\mathbf{P}}
\newcommand{\B}{\mathcal{B}}
\newcommand{\interior}{\text{int}}
\newcommand{\exterior}{\text{ext}}
\newcommand{\bigci}{\mathrel{\text{\scalebox{1.07}{$\perp\mkern-10mu\perp$}}}}
\newcommand{\nsum}{\sum_{n=0}^{\infty}}

\setlist[itemize]{topsep=0pt}
\setlist[enumerate]{topsep=0pt}

\begin{document}

\lecture{3}{}{Steven Heilman}{Kyle Barron}

\subsection*{Exercise 1}
Let $X$ be a random variable. Assume that $M_X(t)$ exists for all $t \in \R$, and assume we can differentiate under the expected value any number of times. For any positive integer $n$, show that 
\[ \left. \frac{d^n}{dt^n} \right|_{t = 0} M_X(t) = \E[X^n]. \]
So, in principle, all moments of $X$ can be computed just by taking derivatives of the moment generating function.





\subsection*{Exercise 2}
Let $X$ be a standard Gaussian random variable. Compute an explicit formula for the moment generating function of $X$. (Hint: completing the square might be helpful.) From this explicit formula, compute an explicit formula for all moments of the Gaussian random variable. (The $2n^{th}$ moment of $X$ should be something resembling a factorial.)










\subsection*{Exercise 3}
Construct two random variables $X$, $Y: \Omega \rightarrow \R$ such that $X \not = Y$ but $M_X(t)$, $M_Y(t)$ exist for all $t \in \R$, and such that $M_X(t) = M_Y(t)$ for all $t \in \R$.








\subsection*{Exercise 4}
(Check expectations notations in original document)

Unfortunately, there exist random variables $X$, $Y$ such that $\E[X]^n = \E[Y]^n$ for all $n = 1, 2, 3, ...$ but such that $X$, $Y$ do not have the same CDF. First, explain why this does not contradict the L\'evy Continuity Theorem, Weak Form. Now, let $-1 < a < 1$, and define a density
\[ f_a(x) \coloneqq \begin{cases}
\frac{1}{x \sqrt{2 \pi}} e^{- \frac{(\log x )^2}{2}} (1 + a \sin(2\pi \log x)), & \text{ if } x > 0 \\
0, & \text{ otherwise.}
\end{cases}
\]
Suppose $X_a$ has density $f_a$. If $-1 < a, b < 1$, show that $\E[X]_a^n = \E[X]_b^n$ for all $n = 1,2,3,...$. (Hint: write out the integrals, and make a change of variables $s = \log(x) - n$.)











































\end{document}
