\documentclass{article}

\setlength{\oddsidemargin}{0 in}
\setlength{\topmargin}{-0.6 in}
\setlength{\textwidth}{6.5 in}
\setlength{\textheight}{8.5 in}
\setlength{\headsep}{0.75 in}
\setlength{\parindent}{0 in}
\setlength{\parskip}{0.1 in}

%

\usepackage{amsmath,amsfonts,graphicx}
\usepackage{amssymb}
\usepackage[utf8]{inputenc}
\usepackage[english]{babel}
\usepackage{enumitem}
\usepackage{graphicx}
\graphicspath{ {images/} }
\usepackage{amsthm}
\usepackage{mathtools}
\usepackage{hyperref}
\usepackage{verbatim}
\usepackage{bbm}
\usepackage{bm}
\usepackage{pgfplots}
\hypersetup{
    colorlinks,
    citecolor=blue,
    filecolor=blue,
    linkcolor=blue,
    urlcolor=blue
}
\usepackage{pgfplots}

%

\makeatletter
\newcommand*\dotp{\mathpalette\dotp@{.5}}
\newcommand*\dotp@[2]{\mathbin{\vcenter{\hbox{\scalebox{#2}{$\m@th#1\bullet$}}}}}
\makeatother

\makeatletter
\newsavebox\myboxA
\newsavebox\myboxB
\newlength\mylenA

\newcommand*\xoverline[2][0.75]{%
    \sbox{\myboxA}{$\m@th#2$}%
    \setbox\myboxB\null% Phantom box
    \ht\myboxB=\ht\myboxA%
    \dp\myboxB=\dp\myboxA%
    \wd\myboxB=#1\wd\myboxA% Scale phantom
    \sbox\myboxB{$\m@th\overline{\copy\myboxB}$}%  Overlined phantom
    \setlength\mylenA{\the\wd\myboxA}%   calc width diff
    \addtolength\mylenA{-\the\wd\myboxB}%
    \ifdim\wd\myboxB<\wd\myboxA%
       \rlap{\hskip 0.5\mylenA\usebox\myboxB}{\usebox\myboxA}%
    \else
        \hskip -0.5\mylenA\rlap{\usebox\myboxA}{\hskip 0.5\mylenA\usebox\myboxB}%
    \fi}
\makeatother

%

\newcounter{lecnum}
\renewcommand{\thepage}{\thelecnum-\arabic{page}}
\renewcommand{\thesection}{\thelecnum.\arabic{section}}
\renewcommand{\theequation}{\thelecnum.\arabic{equation}}
\renewcommand{\thefigure}{\thelecnum.\arabic{figure}}
\renewcommand{\thetable}{\thelecnum.\arabic{table}}

%
% The following macro is used to generate the header.
%
\newcommand{\lecture}[4]{
   \pagestyle{myheadings}
   \thispagestyle{plain}
   \newpage
   \setcounter{lecnum}{#1}
   \setcounter{page}{1}
   \noindent
   \begin{center}
   \framebox{
      \vbox{\vspace{2mm}
    \hbox to 6.28in { {\bf Math 170B
	\hfill Winter 2017} }
       \vspace{4mm}
       \hbox to 6.28in { {\Large \hfill Problem Set #1 \hfill} }
       \vspace{2mm}
       \hbox to 6.28in { {\it Lecturer: #3 \hfill #4} }
      \vspace{2mm}}
   }
   \end{center}
   \markboth{Problem Set #1}{Problem Set #1}
  }

\renewcommand{\cite}[1]{[#1]}
\def\beginrefs{\begin{list}%
        {[\arabic{equation}]}{\usecounter{equation}
         \setlength{\leftmargin}{2.0truecm}\setlength{\labelsep}{0.4truecm}%
         \setlength{\labelwidth}{1.6truecm}}}
\def\endrefs{\end{list}}
\def\bibentry#1{\item[\hbox{[#1]}]}


\newcommand{\fig}[3]{
			\vspace{#2}
			\begin{center}
			Figure \thelecnum.#1:~#3
			\end{center}
	}
% Use these for theorems, lemmas, proofs, etc.
\newtheoremstyle{break}
  {\topsep}{\topsep}%
  {\itshape}{}%
  {\bfseries}{}%
  {\newline}{}%
\theoremstyle{break}
\newtheorem{theorem}{Theorem}[lecnum]
\newtheorem{lemma}[theorem]{Lemma}
\newtheorem{lemma*}{Lemma}
\newtheorem{problem}{Problem}
\newtheorem{proposition}[theorem]{Proposition}
\newtheorem{claim}[theorem]{Claim}
\newtheorem{corollary}[theorem]{Corollary}
\newtheorem{definition}[theorem]{Definition}
\newtheorem{example}{Example}

\newenvironment{solution}{{\bf Solution:}}{\hfill\rule{2mm}{2mm}}
\renewenvironment{proof}{{\bf Proof:}}{\hfill\rule{2mm}{2mm}}

\newcommand{\E}{\mathbb{E}}
\newcommand{\N}{\mathbb{N}}
\newcommand{\R}{\mathbb{R}}
\newcommand{\Z}{\mathbb{Z}}
\newcommand{\Q}{\mathbb{Q}}
\newcommand{\C}{\mathbb{C}}
\newcommand{\X}{\mathbb{X}}
\renewcommand{\L}{\mathcal{L}}
\renewcommand{\P}{\mathbb{P}}
\newcommand{\B}{\mathcal{B}}
\newcommand{\interior}{\text{int}}
\newcommand{\exterior}{\text{ext}}
\newcommand{\bigci}{\mathrel{\text{\scalebox{1.07}{$\perp\mkern-10mu\perp$}}}}
\newcommand{\nsum}{\sum_{n=0}^{\infty}}

\setlist[itemize]{topsep=0pt}
\setlist[enumerate]{topsep=0pt}

\begin{document}

\lecture{1}{}{Steven Heilman}{Kyle Barron}


\subsection*{Exercise 1}
Using the De Moivre-Laplace Theorem, estimate the probability that 1,000,000 coin flips of fair coins will result in more than 501,000 heads. (Some of the following integrals may be relevant: $\int_{-\infty}^0 e^{-t^2/2} dt/\sqrt{2 \pi} = 1/2$, $\int_{-\infty}^1 e^{-t^2/2} dt/\sqrt{2 \pi} \approx .8413$, $\int_{-\infty}^2 e^{-t^2/2} dt/\sqrt{2 \pi} \approx .9772$, $\int_{-\infty}^3 e^{-t^2/2} dt/\sqrt{2 \pi} \approx .9987$.)

\begin{solution}
Since coin flips of fair coins are independent Bernoulli random variables, we can use the De Moivre-Laplace Theorem. Let $n =$ 1,000,000.
\begin{align*}
\P(X_1 + ... + X_n > 501000) &= 1 - \P( X_1 + ... + X_n \leq 501000) \\
	&= 1 - \P( X_1 + ... + X_n - 500000 \leq 1000) \\
	&= 1 - \P \left( \frac{X_1 + ... + X_n - 500000}{1000} \leq 1 \right) \\ 
	&= 1 - \P \left( \frac{X_1 + ... + X_n - 500000}{1000 \cdot \sqrt{1/4}} \leq 2 \right) \\
	&= 1 - \P \left( \frac{X_1 + ... + X_n - (1/2)n}{\sqrt{n} \cdot \sqrt{1/4}} \leq 2 \right) \\
	&\approx 1 - \int_{-\infty}^2 e^{-t^2/2} \frac{dt}{\sqrt{2\pi}} \\
	&= 1 - .9772 = .0228.
\end{align*}
(Where the approximation is given by the De Moivre-Laplace Theorem, and $a = 2$.)
\end{solution}


\subsection*{Exercise 2}
Let $X$ and $Y$ be nonnegative random variables. Recall that we can define 
\[ \E[X] \coloneqq \int_{0}^\infty \P(X > t) dt.
\]
Assume that $X \leq Y$. Conclude that $\E[X] \leq \E[Y]$. 

More generally, if $X$ satisfies $\E[ |X| ] < \infty$, we define $\E[X] \coloneqq \E [ \max(X,0)] - \E[ \max(-X,0)]$. If $X, Y$, are any random variables with $X \leq Y$, $\E[|X|] < \infty$ and $\E[ |Y| ] < \infty$, show that $\E[X] \leq \E[Y]$. 

\begin{solution}
Suppose that $X \leq Y$. We want to show that $\E[X] \leq \E[Y]$. Well by the definition of $\E[X]$ above, we can write
\begin{align*}
\E[X] \leq \E[Y] &\Leftarrow \int_0^\infty \P(X > t) dt \leq \int_0^\infty \P(Y > t) dt \\
	&\Leftarrow \P(X > t) \leq \P(Y > t), \forall t \\
	&\Leftarrow \{ X > t \} \subseteq \{ Y > t \}, \forall t \\
	&\Leftarrow \text{ if } X > t, Y \geq X > t, \forall t \\
	&\Leftarrow X \leq Y.
\end{align*}
That is, since $X \leq Y$, then if $X > t$, then $Y \geq X > t$. That means that the set of all $X$ that is greater than $t$ is contained within the set of all $Y$ that is greater than $t$. Hence $\P(X > t) \leq \P(Y > t), \forall t$. Thus $\E[X] \leq \E[Y]$.

\end{solution}

\subsection*{Exercise 3}
Using the definition of convergence, show that the sequence of numbers 
\[ 1, 1/2, 1/3, 1/4, ... \]
converges to 0. 

\begin{proof}
Let $x_n = \frac{1}{n}$, let $\epsilon > 0$, and let $M = \frac{1}{\epsilon}$. Then for $n > M$ we have 
\[ |x_n - x| = \left| \frac{1}{n} - 0 \right| = \frac{1}{n} < \frac{1}{M} = \epsilon.\]
Therefore the sequence $(x_n)$ with $x_n = \frac{1}{n}$ converges to 0.
\end{proof}

\subsection*{Exercise 4}
Let $x_1, x_2, ...$ be a sequence of real numbers. Let $x,y \in \R$. Assume that $x_1, x_2, ...$ converges to $x$. Assume also that $x_1, x_2, ... $ converges to $y$. Prove that $x = y$. That is, a sequence of real numbers cannot converge to two different real numbers.

\begin{proof}
Call $(x_n)$ the sequence of $x_1, x_2, ...$. Let $x,y \in \R$. Suppose that $x_n \rightarrow x$ and $x_n \rightarrow y$. Suppose for contradiction that $x \not = y$. 

Let $\epsilon = \frac{|x-y|}{3}$. Since $x_n \rightarrow x$, there exists $M_x$ such that $n > M_x \Rightarrow |x_n - x| < \epsilon$. Since $x_n \rightarrow y$, there exists $M_y$ such that $n > M_y \Rightarrow |x_n - y| < \epsilon$. Let $Z = \max\{M_x, M_y\}$. Then when $n > Z$,
\[ |x - y| = |x - x_n + x_n - y| \leq |x_n - x| + |x_n - y| < \frac{|x-y|}{3} + \frac{|x-y|}{3} < |x - y|,
\]
where the second relation is from the triangle inequality and the third is using the convergence of $x_n$ to $x$ and $y$. This is a contradiction so $x = y$.
\end{proof}


\subsection*{Exercise 5}
Let $X$ be a uniformly distributed random variable on $[-1,1]$. Let $Y \coloneqq X^2$. Find $f_Y$. 

\begin{solution}
Let $X$ be uniformly distributed on $[-1,1]$, let $g(x) = x^2$, and let $Y = g(X)$. Since $X$ is continuous, $x^2$ is continuous and $F_Y$ is differentiable we can use Proposition 2.6 from the notes. 

First note that
\[ f_X(x) = \begin{cases}
\frac{1}{2}, & x \in [-1,1] \\
0, & \text{otherwise}.
\end{cases}
\]

Then
\begin{align*}
f_Y(y) &= \frac{d}{dy} \int_{\{x \in \R : g(x) \leq y\}} f_X(x)dx \\
&= \frac{d}{dy} \int_{\{x \in [-1,1]: x^2 \leq y \}} \frac{1}{2} dx.
\end{align*}
If $y < 0$, the integral is zero. If $y > 1$ the integral is 1. If $y \in [0,1]$ we have 
\begin{align*}
f_Y(y) &= \frac{d}{dy} \frac{1}{2} \int_{-y^{1/2}}^{y^{1/2}} dx \\
&= \frac{1}{2} \frac{d}{dy} [y^{1/2} + y^{1/2}] \\
&= \frac{1}{2 \sqrt{y}}. 
\end{align*}
So the PDF of $Y$ is 
\[ f_Y = \begin{cases}
0, & y < 0 \\
\frac{1}{2 \sqrt{y}}, & y \in [0,1] \\
0, & y > 1.
\end{cases}
\]
\end{solution}

\subsection*{Exercise 6}
Let $X$ be a uniformly distributed random variable on $[0,1]$. Let $Y \coloneqq 4X(1-X)$. Find $f_Y$.

\begin{solution}
We wish to find $f_Y$. We'll first find $F_Y$ and then take the derivative. First note that when $x \in [0,1]$, the image of $4x(1-x)$ is $[0,1]$. So we only have to deal with $y$ values between $0$ and $1$. Note that $f_X(x) = 1$ on $x \in [0,1]$. 
\begin{align*}
\P(Y \leq y) &= \P(4X(1-X) \leq y) \\
	&= \P \left( 0 \leq x \leq \frac{1 - \sqrt{1-y}}{2} \right) + \P \left(\frac{1 + \sqrt{1-y}}{2} \leq x \leq 1 \right) \\
	&= \frac{1 - \sqrt{1-y}}{2} + 1 - \frac{1 + \sqrt{1-y}}{2} \\
	&= 1 - \sqrt{1-y} \text{ for } y \in [0,1].
\end{align*}
Therefore, the PDF of $Y$ is:
\[ f_Y(y) = F_Y'(y) = \begin{cases}
\frac{1}{\sqrt{1-y}}, & y \in [0,1] \\
0, & \text{ otherwise}.
\end{cases} 
\]


\begin{comment}
Let $X$ be uniformly distributed on $[0,1]$, let $g(x) = 4x(1-x)$, and let $Y = g(X)$. Since $X$ is continuous, $g(x)$ is continuous, and assuming $F_Y$ is differentiable, we can again use Proposition 2.6 from the notes. 

Since $X$ is uniformly distributed on $[0,1]$, $f_X = 1$ for $x \in [0,1]$ and $f_X = 0$ otherwise. Then \begin{align*}
F_Y(y) &= \frac{d}{dy} \int_{\{x \in \R : g(x) \leq y\}} f_X(x)dx \\
&= \frac{d}{dy} \int_{\{x \in [0,1]: 4x(1-x) \leq y \}} 1  dx.
\end{align*}
If $y < 0$, the integral is zero. If $y > 1$, the integral is 1. If $y \in [0, 1]$, we have
\begin{align*}
f_Y(y) &= \frac{d}{dy} \int_{\{x \in [0,1]: 4x(1-x) \leq y \}} 1  dx \\
	&= \frac{d}{dy} \left[ 1 -  \int_{\{x \in [0,1]: 4x(1-x) > y \}} 1 dx \right] \\
	&= \frac{d}{dy} \left[ 1 -  \int_{\{x \in [0,1]: 4x(1-x) > y \}} 1 dx \right] 
\end{align*}
\end{comment}





\end{solution}

\subsection*{Exercise 7}
Let $X$ be a uniformly distributed random variable on $[0,1]$. Find the PDF of $-\log(X)$.


\begin{solution}

Recall that the CDF of $X$ is $F_X(x) = \begin{cases}
0, & x \leq 0 \\
x, & x \in [0,1] \\
1, & x \geq 1.
\end{cases}$.

Then to find the CDF of $Y$,
\[ F_Y(y) = \P(Y \leq y) = \P(- \log(X) \leq y) = \P(X \geq e^{-y}) = 1 - \P(X \leq e^{-y}) = 1 - e^{-y}.
\]
Therefore $F_Y(y) = \begin{cases}
0, & y < 0 \\
1 - e^{-y}, & y \geq 0
\end{cases}
$, so $f_Y(y) = \begin{cases}
0, & y < 0 \\
e^{-y}, & y \geq 0
\end{cases}
$.



\end{solution}


\subsection*{Exercise 8}
Let $X$ be a standard normal random variable. Find the PDF of $e^X$. 

\begin{solution}
Let $X$ be standard normal. Let $Y = e^X$.
\begin{align*}
f_Y(y) &= \frac{d}{dy} \int_{\{x \in \R : g(x) \leq y \}} \frac{1}{\sqrt{2 \pi}} e ^{-\frac{x^2}{2}} dx \\
	&= \frac{d}{dy} \int_{\{x \in \R : e^x \leq y \}} \frac{1}{\sqrt{2 \pi}} e ^{-\frac{x^2}{2}} dx \\
	 &= \frac{d}{dy} \int_{\{x \in \R : x \leq \ln(y) \}} \frac{1}{\sqrt{2 \pi}} e ^{-\frac{x^2}{2}} dx \\
	  &= \frac{d}{dy} \int_{\infty}^{\ln(y)} \frac{1}{\sqrt{2 \pi}} e ^{-\frac{x^2}{2}} dx \\
	  &= \frac{1}{\sqrt{2 \pi}} e ^{- \frac{\ln(y)^2}{2}} \\
	  &= \frac{1}{\sqrt{2 \pi}} e ^{ \ln (y^{-1/2})^2} \\
	  &= \frac{1}{\sqrt{2 \pi}} \left( e ^{ \ln (y^{-1/2})} \right)^{ \ln (y^{-1/2})} \\
	  &= \frac{1}{\sqrt{2 \pi}} \left( y^{-1/2} \right)^{ \ln (y^{-1/2})}.
\end{align*}
Note that $f_Y(y)$ is defined this way on $y \in \R$.
\end{solution}

\subsection*{Exercise 9}
Let $X,Y,Z$ be independent standard Gaussian random variables. Find the PDF of $\max(X,Y,Z)$.

\begin{solution}
Let $X, Y, Z$ be independent standard Gaussian random variables. Denote $W = \max\{X,Y,Z\}$.
First I'll find the CDF of $W$ and then I'll take the derivative to get the PDF of $W$.
\begin{align*}
F_W(w) &= \P(W \leq w) \\
	&= \P( X \leq w, Y \leq w, Z \leq w) \\
	&= \P(X \leq w) \P(Y \leq w) \P(Z \leq w) \\
	&= F_X(w) F_Y(w) F_Z(w) \\
	&= F_X(w)^3. \\
f_W(w) &= 3F_X(w)^2 \cdot f_X(w).	
\end{align*}
Where $f_X(w) = \frac{1}{\sqrt{2\pi}} e^{-w^2/2}$ and $F_X(w) = \int_{-\infty}^w \frac{1}{\sqrt{2\pi}} e^{-t^2/2} dt$.

\end{solution}


\subsection*{Exercise 10}
Let $X$ be a random variable uniformly distributed in $[0,1]$ and let $Y$ be a random variable uniformly distributed in $[0,2]$. Suppose $X$ and $Y$ are independent. Find the PDF of $X/Y^2$.


\begin{solution}
Let $X$ be uniformly distributed on $[0,1]$ and let $Y$ be distributed uniformly on $[0,2]$. Let $Z = \frac{X}{Y^2}$. First I'll find the CDF of $Z$ and then I'll find the PDF of $Z$. Since $X$ and $Y$ are both positive, the range for $Z$ is $[0, \infty)$.

To find the CDF of $Z$ we need to do 
\[ \P(Z \leq z) = \P\left(\frac{X}{Y^2} \leq z\right) = \int \int_{\{0 \leq x \leq 1, 0 \leq y \leq 2, \frac{x}{y^2} \leq z \}} f_{X,Y}(x,y) dx dy = \int \int_{\{0 \leq x \leq 1, 0 \leq y \leq 2, \frac{x}{y^2} \leq z \}} f_X(x) f_Y(y) dx dy
\]
where the last equality is because $X$ and $Y$ are independent. The substituting in the two PDF's, the integral becomes $\frac{1}{2} \int \int dx dy$.

We now need to break this up into cases. 
\begin{itemize}
\item Case 1: $z \cdot y^2 \geq 1$. Then $y \geq \sqrt{\frac{1}{z}} \Rightarrow 0 \leq X \leq 1$.
\item Case 2: $z \cdot y^2 < 1$. Then $y < \sqrt{\frac{1}{z}} \Rightarrow 0 \leq x \leq z \cdot y^2$.
\end{itemize}
Those are the cases for $z$ and $y$ jointly but we also have the cases just depending on $z$:
\begin{itemize}
\item Case A: if $\sqrt{\frac{1}{z}} \geq 2$, only case 2 exists above.

\begin{align*}
F_Z(z) &= \frac{1}{2} \left( \int_0^2 \int_0^{zy^2} dx dy \right) \\
	&= \frac{1}{2} \left( \int_0^2 zy^2 dy \right) \\
	&= \frac{1}{2} \left[ \frac{zy^3}{3} \right]_{y = 0}^{y = 2} \\
	&= \frac{4z}{3}, \text{ for } 0 \leq z \leq \frac{1}{4}.
\end{align*}

\item Case B: if $\sqrt{\frac{1}{z}} < 2$, both case 1 and 2 above exist. 
\begin{align*}
F_Z(z) &= \frac{1}{2} \left( \int_{\sqrt{\frac{1}{z}}}^2 \int_0^1 dx dy + \int_0^{\sqrt{\frac{1}{z}}} \int_0^{zy^2} dx dy \right) \\
	&= \frac{1}{2} \left( 2 - \sqrt{\frac{1}{z}} + \left[ \frac{zy^3}{3} \right]_{y = 0}^{y = {\sqrt{\frac{1}{z}}}}  \right) \\
	&= \frac{1}{2} \left( 2 - \sqrt{\frac{1}{z}} + \frac{z \cdot z^{-3/2}}{3} \right) \\
	&= \frac{1}{2} \left( 2 - \sqrt{\frac{1}{z}} + \frac{z^{-1/2}}{3} \right) \\
	&= \frac{3 \sqrt{z} - 1}{3 \sqrt{z}}, \text{ for } z > \frac{1}{4}.
\end{align*}
\end{itemize}
So therefore, taking the derivative of each we get
\[ f_Z(z) = \begin{cases}
\frac{1}{6 z^{3/2}}, & z > \frac{1}{4} \\
\frac{4}{3}, & 0 \leq z \leq \frac{1}{4}.
\end{cases}
\]

\end{solution}





























\end{document}
