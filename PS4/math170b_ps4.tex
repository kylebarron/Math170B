\documentclass{article}

\setlength{\oddsidemargin}{0 in}
\setlength{\topmargin}{-0.6 in}
\setlength{\textwidth}{6.5 in}
\setlength{\textheight}{8.5 in}
\setlength{\headsep}{0.75 in}
\setlength{\parindent}{0 in}
\setlength{\parskip}{0.1 in}

%

\usepackage{amsmath,amsfonts,graphicx}
\usepackage{amssymb}
\usepackage[utf8]{inputenc}
\usepackage[english]{babel}
\usepackage{enumitem}
\usepackage{graphicx}
\graphicspath{ {images/} }
\usepackage{amsthm}
\usepackage{mathtools}
\usepackage{hyperref}
\usepackage{verbatim}
\usepackage{bbm}
\usepackage{bm}
\usepackage{pgfplots}
\hypersetup{
    colorlinks,
    citecolor=blue,
    filecolor=blue,
    linkcolor=blue,
    urlcolor=blue
}
\usepackage{pgfplots}

%

\makeatletter
\newcommand*\dotp{\mathpalette\dotp@{.5}}
\newcommand*\dotp@[2]{\mathbin{\vcenter{\hbox{\scalebox{#2}{$\m@th#1\bullet$}}}}}
\makeatother

\makeatletter
\newsavebox\myboxA
\newsavebox\myboxB
\newlength\mylenA

\newcommand*\xoverline[2][0.75]{%
    \sbox{\myboxA}{$\m@th#2$}%
    \setbox\myboxB\null% Phantom box
    \ht\myboxB=\ht\myboxA%
    \dp\myboxB=\dp\myboxA%
    \wd\myboxB=#1\wd\myboxA% Scale phantom
    \sbox\myboxB{$\m@th\overline{\copy\myboxB}$}%  Overlined phantom
    \setlength\mylenA{\the\wd\myboxA}%   calc width diff
    \addtolength\mylenA{-\the\wd\myboxB}%
    \ifdim\wd\myboxB<\wd\myboxA%
       \rlap{\hskip 0.5\mylenA\usebox\myboxB}{\usebox\myboxA}%
    \else
        \hskip -0.5\mylenA\rlap{\usebox\myboxA}{\hskip 0.5\mylenA\usebox\myboxB}%
    \fi}
\makeatother

%

\newcounter{lecnum}
\renewcommand{\thepage}{\thelecnum-\arabic{page}}
\renewcommand{\thesection}{\thelecnum.\arabic{section}}
\renewcommand{\theequation}{\thelecnum.\arabic{equation}}
\renewcommand{\thefigure}{\thelecnum.\arabic{figure}}
\renewcommand{\thetable}{\thelecnum.\arabic{table}}

%
% The following macro is used to generate the header.
%
\newcommand{\lecture}[4]{
   \pagestyle{myheadings}
   \thispagestyle{plain}
   \newpage
   \setcounter{lecnum}{#1}
   \setcounter{page}{1}
   \noindent
   \begin{center}
   \framebox{
      \vbox{\vspace{2mm}
    \hbox to 6.28in { {\bf Math 170B
	\hfill Winter 2017} }
       \vspace{4mm}
       \hbox to 6.28in { {\Large \hfill Problem Set #1 \hfill} }
       \vspace{2mm}
       \hbox to 6.28in { {\it Lecturer: #3 \hfill #4} }
      \vspace{2mm}}
   }
   \end{center}
   \markboth{Problem Set #1}{Problem Set #1}
  }

\renewcommand{\cite}[1]{[#1]}
\def\beginrefs{\begin{list}%
        {[\arabic{equation}]}{\usecounter{equation}
         \setlength{\leftmargin}{2.0truecm}\setlength{\labelsep}{0.4truecm}%
         \setlength{\labelwidth}{1.6truecm}}}
\def\endrefs{\end{list}}
\def\bibentry#1{\item[\hbox{[#1]}]}


\newcommand{\fig}[3]{
			\vspace{#2}
			\begin{center}
			Figure \thelecnum.#1:~#3
			\end{center}
	}
% Use these for theorems, lemmas, proofs, etc.
\newtheoremstyle{break}
  {\topsep}{\topsep}%
  {\itshape}{}%
  {\bfseries}{}%
  {\newline}{}%
\theoremstyle{break}
\newtheorem{theorem}{Theorem}[lecnum]
\newtheorem{lemma}[theorem]{Lemma}
\newtheorem{lemma*}{Lemma}
\newtheorem{problem}{Problem}
\newtheorem{proposition}[theorem]{Proposition}
\newtheorem{claim}[theorem]{Claim}
\newtheorem{corollary}[theorem]{Corollary}
\newtheorem{definition}[theorem]{Definition}
\newtheorem{example}{Example}

\newenvironment{solution}{{\bf Solution:}}{\hfill\rule{2mm}{2mm}}
\renewenvironment{proof}{{\bf Proof:}}{\hfill\rule{2mm}{2mm}}

\newcommand{\E}{\mathrm{E}}
\newcommand{\var}{\mathrm{Var}}
\newcommand{\cov}{\mathrm{Cov}}
\newcommand{\N}{\mathbb{N}}
\newcommand{\R}{\mathbb{R}}
\newcommand{\Z}{\mathbb{Z}}
\newcommand{\Q}{\mathbb{Q}}
\newcommand{\C}{\mathbb{C}}
\newcommand{\X}{\mathbb{X}}
\renewcommand{\L}{\mathcal{L}}
\renewcommand{\P}{\mathbf{P}}
\newcommand{\B}{\mathcal{B}}
\newcommand{\interior}{\text{int}}
\newcommand{\exterior}{\text{ext}}
\newcommand{\bigci}{\mathrel{\text{\scalebox{1.07}{$\perp\mkern-10mu\perp$}}}}
\newcommand{\nsum}{\sum_{n=0}^{\infty}}

\setlist[itemize]{topsep=0pt}
\setlist[enumerate]{topsep=0pt}

\begin{document}

\lecture{4}{}{Steven Heilman}{Kyle Barron}

\section*{Exercise 1}

Compute the characteristic function of a uniformly distributed random variable on $[-1,1]$.  (Some of the following formulas might help to simplify your answer: $e^{it}=\cos(t)+i\sin (t)$, $\cos(t)=[e^{it}+e^{-it}]/2$, $\sin(t)=[e^{it}-e^{-it}]/[2i]$, $t\in\R$.)  (Here $i\coloneqq \sqrt{-1}$.)

\begin{solution}
\begin{align*}
\phi_X(t) &= \E[e^{itx}] \\
	&= \frac{1}{2} \int_{-1}^1 e^{itx} dx \\
	&= \left. \frac{1}{2 i t} e^{itx} \right|_{-1}^1 \\
	&= \frac{1}{2it} \left( e^{it} - e^{-it} \right) \\
	&= \frac{\sin(t)}{t}.
\end{align*}
\end{solution}

\section*{Exercise 2}

Let $X$ be a random variable.  Assume we can differentiate under the expected value of $\E e^{itX}$ any number of times.  For any positive integer $n$, show that
\[ \left. \frac{d^n}{dt^n} \phi_{X}(t) \right|_{t=0}=i^{n}\E(X^{n}).\]
So, in principle, all moments of $X$ can be computed just by taking derivatives of the characteristic function.

\begin{proof}

\begin{align*}
\frac{d^n}{dt^n} \phi_X(t) &= \frac{d^n}{dt^n} \int_{-\infty}^\infty e^{itx} f_X(x) dx\\
	&= \int_{-\infty}^\infty \frac{d^n}{dt^n} e^{itx} f_X(x) dx \\
	&= \int_{-\infty}^\infty (ix)^n e^{itx} f_X(x) dx.
\end{align*}
Now considering the special case where $t = 0$, we get that 
\begin{align*}
\left. \frac{d^n}{dt^n} \phi_X(t) \right|_{t=0} &= \int_{-\infty}^\infty (ix)^n f_X(x) dx \\
	&= i^n \int_{-\infty}^\infty x^n f_X(x) dx \\
	&= i^n \E[X^n]
\end{align*}
\end{proof}

\section*{Exercise 3}
Let $X$ be a random variable such that $\E[ |X|^{3}] <\infty$.  Prove that for any $t\in\R$,
\begin{align*}
\E e^{itX}=1+it\E X-t^{2}\E X^{2}/2+o(t^{2}).
\end{align*}
That is,
$$\lim_{t \rightarrow 0}t^{-2} |\E e^{itX}-[1+it\E X-t^{2}\E X^{2}/2]|=0.$$
(Hint: it may be helpful to use Jensen's inequality to first justify that $\E |X|<\infty$ and $\E X^{2}<\infty$.  Then, use the Taylor expansion with error bound: $e^{iy}=1+iy-y^{2}/2-(i/2)\int_{0}^{y}(y-s)^{2}e^{is}ds$, which is valid for any $y\in\R$.)

Actually, this same bound holds only assuming $\E X^{2}<\infty$, but the proof of that bound requires things we have not discussed.

\begin{solution}
Let $X$ be a random variable with $\E[|X|^3] < \infty$. I'll first show that $\E[|X|] < \infty$. Let $f(X) = |X|^3$; note that this is a convex function. 
Then $\E[f(X)] = \E[|X|^3]$ and $f(\E[X]) = |\E[X]|^3 = \E[X]^3$. We can now use Jensen's inequality. 
\[ \infty > \E[|X|^3] \geq |\E[X]|^3 = \E[X]^3.\]
Therefore, $\E[X] < \infty$.



\end{solution}


\section*{Exercise 4}
(Convolution is Associative.)
Let $g,h,d\colon\R\to\R$.  Then for any $t\in\R$,
$$((g*h)*d)(t)=(g*(h*d))(t)$$


\begin{proof}

\begin{align*}
((g \ast h) \ast d)(t)&= \int_{-\infty}^\infty (g \ast h)(x) d(t-x) dx \\
	&= \int_{-\infty}^\infty \int_{-\infty}^\infty (g(s) h(x-s) ds ) d(t-x) dx \\
	&= \int_{-\infty}^\infty \int_{-\infty}^\infty g(s) h(x-s) d(t-x) ds dx
\end{align*}

\end{proof}

\section*{Exercise 5}
Let $X,Y,Z$ be independent and uniformly distributed on $[0,1]$.  Note that $f_{X}$ is not a continuous function.

Using convolution, compute $f_{X+Y}$.  Draw $f_{X+Y}$.  Note that $f_{X+Y}$ is a continuous function, but it is not differentiable at some points.

Using convolution, compute $f_{X+Y+Z}$.  Draw $f_{X+Y+Z}$.  Note that $f_{X+Y+Z}$ is a differentiable function, but it does not have a second derivative at some points.

Make a conjecture about how many derivatives $f_{X_{1}+\cdots+X_{n}}$ has, where $X_{1},\ldots,X_{n}$ are independent and uniformly distributed on $[0,1]$.  You do not have to prove this conjecture.  The idea of this exercise is that convolution is a kind of average of functions.  And the more averaging you do, the more derivatives $f_{X_{1}+\cdots+X_{n}}$ has.

\begin{solution}
Let $X,Y,Z$ be independent and uniformly distributed over $[0,1]$. First I'll compute $f_{X+Y}$. Recall that Proposition 2.60 says that $f_{X+Y}(t) = (f_X \ast f_Y)(t)$, for all $t \in \R$. Recall that by Definition 2.59, $(g \ast h)(t) \coloneqq \int_{-\infty}^\infty g(x) h(t-x) dx$, for all $t \in \R$. Therefore,
\begin{align*}
f_{X+Y}(t) &= (f_X \ast f_Y)(t) \\
	&= \int_{-\infty}^\infty f_X(x) f_Y(t-x) dx \\
	&= \int_0^1 1 \cdot f_Y(t-x) dx \\
	&= \int_{x \in [0,1]\cap[t-1,t]} 1 dx
\end{align*}





\end{solution}


\section*{Exercise 6}
Construct two random variables $X,Y$ such that $X$ and $Y$ are each uniformly distributed on $[0,1]$, and such that $\P(X+Y=1)=1$.
% Y=1-X

Then construct two random variables $W,Z$ such that $W$ and $Z$ are each uniformly distributed on $[0,1]$, and such that $W+Z$ is uniformly distributed on $[0,2]$.
%X=Y

(Hint: there is a way to do each of the above problems with about one line of work.  That is, there is a way to solve each problem without working very hard.)

\begin{solution}
For the first part, let $X$ be uniformly distributed on $[0,1]$ with $f_X(x) = 1$ for $x \in [0,1]$. Let $Y = 1 - X$. Then $Y$ is also distributed uniformly on $[0,1]$ and $X + Y = X + 1 - X = 1$. 

For the second part, let $W$ be distributed as $f_W(x) = 1$ for $x \in [0,1]$. Let $Z = W$. Then 
\[ \P(W + Z = t) = \P(2W = t) = \P\left(W = \frac{t}{2}\right).
\]
So $f_{W+Z}(t) = \frac{1}{2}$ for $t \in (0,2)$.
\begin{comment}
\begin{align*}
f_W(x) &= \P(W = x) \\
	&= \P(2W = 2x) \\
	&= \P(W + Z = 2x)
\end{align*}
\end{comment}
\end{solution}











































































\end{document}
